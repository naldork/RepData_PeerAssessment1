\documentclass[]{article}
\usepackage{lmodern}
\usepackage{amssymb,amsmath}
\usepackage{ifxetex,ifluatex}
\usepackage{fixltx2e} % provides \textsubscript
\ifnum 0\ifxetex 1\fi\ifluatex 1\fi=0 % if pdftex
  \usepackage[T1]{fontenc}
  \usepackage[utf8]{inputenc}
\else % if luatex or xelatex
  \ifxetex
    \usepackage{mathspec}
  \else
    \usepackage{fontspec}
  \fi
  \defaultfontfeatures{Ligatures=TeX,Scale=MatchLowercase}
\fi
% use upquote if available, for straight quotes in verbatim environments
\IfFileExists{upquote.sty}{\usepackage{upquote}}{}
% use microtype if available
\IfFileExists{microtype.sty}{%
\usepackage[]{microtype}
\UseMicrotypeSet[protrusion]{basicmath} % disable protrusion for tt fonts
}{}
\PassOptionsToPackage{hyphens}{url} % url is loaded by hyperref
\usepackage[unicode=true]{hyperref}
\hypersetup{
            pdftitle={Reproducible Research: Peer Assessment 1},
            pdfborder={0 0 0},
            breaklinks=true}
\urlstyle{same}  % don't use monospace font for urls
\usepackage[margin=1in]{geometry}
\usepackage{graphicx,grffile}
\makeatletter
\def\maxwidth{\ifdim\Gin@nat@width>\linewidth\linewidth\else\Gin@nat@width\fi}
\def\maxheight{\ifdim\Gin@nat@height>\textheight\textheight\else\Gin@nat@height\fi}
\makeatother
% Scale images if necessary, so that they will not overflow the page
% margins by default, and it is still possible to overwrite the defaults
% using explicit options in \includegraphics[width, height, ...]{}
\setkeys{Gin}{width=\maxwidth,height=\maxheight,keepaspectratio}
\IfFileExists{parskip.sty}{%
\usepackage{parskip}
}{% else
\setlength{\parindent}{0pt}
\setlength{\parskip}{6pt plus 2pt minus 1pt}
}
\setlength{\emergencystretch}{3em}  % prevent overfull lines
\providecommand{\tightlist}{%
  \setlength{\itemsep}{0pt}\setlength{\parskip}{0pt}}
\setcounter{secnumdepth}{0}
% Redefines (sub)paragraphs to behave more like sections
\ifx\paragraph\undefined\else
\let\oldparagraph\paragraph
\renewcommand{\paragraph}[1]{\oldparagraph{#1}\mbox{}}
\fi
\ifx\subparagraph\undefined\else
\let\oldsubparagraph\subparagraph
\renewcommand{\subparagraph}[1]{\oldsubparagraph{#1}\mbox{}}
\fi

% set default figure placement to htbp
\makeatletter
\def\fps@figure{htbp}
\makeatother


\title{Reproducible Research: Peer Assessment 1}
\author{}
\date{\vspace{-2.5em}}

\begin{document}
\maketitle

\subsection{Loading and preprocessing the
data}\label{loading-and-preprocessing-the-data}

Loading the Data

library(ggplot2) \#\# Warning: package `ggplot2' was built under R
version 3.1.3 library(plyr) \#\# Warning: package `plyr' was built under
R version 3.1.3 activity \textless{}- read.csv(``activity.csv'')
Processing the Data

activity\(day <- weekdays(as.Date(activity\)date))
activity\(DateTime<- as.POSIXct(activity\)date, format=``\%Y-\%m-\%d'')

\subsection{pulling data without nas}\label{pulling-data-without-nas}

clean \textless{}- activity{[}!is.na(activity\$steps),{]}

\subsection{What is mean total number of steps taken per
day?}\label{what-is-mean-total-number-of-steps-taken-per-day}

Calculate the total number of steps taken per day

\subsection{summarizing total steps per
date}\label{summarizing-total-steps-per-date}

sumTable \textless{}- aggregate(activity\(steps ~ activity\)date,
FUN=sum, ) colnames(sumTable)\textless{}- c(``Date'', ``Steps'') Make a
histogram of the total number of steps taken each day

\subsection{Creating the historgram of total steps per
day}\label{creating-the-historgram-of-total-steps-per-day}

hist(sumTable\(Steps, breaks=5, xlab="Steps", main = "Total Steps per Day") ## Mean of Steps as.integer(mean(sumTable\)Steps))
\#\# {[}1{]} 10766 \#\# Median of Steps
as.integer(median(sumTable\$Steps)) \#\# {[}1{]} 10765 The average
number of steps taken each day was 10766 steps.

The median number of steps taken each day was 10765 steps.

\subsection{What is the average daily activity
pattern?}\label{what-is-the-average-daily-activity-pattern}

ake a time series plot (i.e.~type = ``l'') of the 5-minute interval
(x-axis) and the average number of steps taken, averaged across all days
(y-axis)

library(plyr) library(ggplot2) \#\#pulling data without nas clean
\textless{}- activity{[}!is.na(activity\$steps),{]}

\subsection{create average number of steps per
interval}\label{create-average-number-of-steps-per-interval}

intervalTable \textless{}- ddply(clean, .(interval), summarize, Avg =
mean(steps))

\subsection{Create line plot of average number of steps per
interval}\label{create-line-plot-of-average-number-of-steps-per-interval}

p \textless{}- ggplot(intervalTable, aes(x=interval, y=Avg), xlab =
``Interval'', ylab=``Average Number of Steps'') p +
geom\_line()+xlab(``Interval'')+ylab(``Average Number of
Steps'')+ggtitle(``Average Number of Steps per Interval'')

Which 5-minute interval, on average across all the days in the dataset,
contains the maximum number of steps?

\subsection{Maximum steps by interval}\label{maximum-steps-by-interval}

maxSteps \textless{}-
max(intervalTable\(Avg) ##Which interval contains the maximum average number of steps intervalTable[intervalTable\)Avg==maxSteps,1{]}
\#\# {[}1{]} 835 The maximum number of steps for a 5-minute interval was
206 steps.

The 5-minute interval which had the maximum number of steps was the 835
interval.

\subsection{Imputing missing values}\label{imputing-missing-values}

Calculate and report the total number of missing values in the dataset
(i.e.~the total number of rows with NAs)

\subsection{Number of NAs in original data
set}\label{number-of-nas-in-original-data-set}

nrow(activity{[}is.na(activity\$steps),{]}) \#\# {[}1{]} 2304 The total
number of rows with steps = `NA' is 2304.

Devise a strategy for filling in all of the missing values in the
dataset. The strategy does not need to be sophisticated. For example,
you could use the mean/median for that day, or the mean for that
5-minute interval, etc.

My strategy for filling in NAs will be to substitute the missing steps
with the average 5-minute interval based on the day of the week.

\subsection{Create the average number of steps per weekday and
interval}\label{create-the-average-number-of-steps-per-weekday-and-interval}

avgTable \textless{}- ddply(clean, .(interval, day), summarize, Avg =
mean(steps))

\subsection{Create dataset with all NAs for
substitution}\label{create-dataset-with-all-nas-for-substitution}

nadata\textless{}- activity{[}is.na(activity\$steps),{]} \#\# Merge NA
data with average weekday interval for substitution
newdata\textless{}-merge(nadata, avgTable, by=c(``interval'', ``day''))
Create a new dataset that is equal to the original dataset but with the
missing data filled in.

\subsection{Reorder the new substituded data in the same format as clean
data
set}\label{reorder-the-new-substituded-data-in-the-same-format-as-clean-data-set}

newdata2\textless{}- newdata{[},c(6,4,1,2,5){]}
colnames(newdata2)\textless{}- c(``steps'', ``date'', ``interval'',
``day'', ``DateTime'')

\subsection{Merge the NA averages and non NA data
together}\label{merge-the-na-averages-and-non-na-data-together}

mergeData \textless{}- rbind(clean, newdata2) Make a histogram of the
total number of steps taken each day and Calculate and report the mean
and median total number of steps taken per day. Do these values differ
from the estimates from the first part of the assignment? What is the
impact of imputing missing data on the estimates of the total daily
number of steps?

\subsection{Create sum of steps per date to compare with step
1}\label{create-sum-of-steps-per-date-to-compare-with-step-1}

sumTable2 \textless{}- aggregate(mergeData\(steps ~ mergeData\)date,
FUN=sum, ) colnames(sumTable2)\textless{}- c(``Date'', ``Steps'')

\subsection{Mean of Steps with NA data taken care
of}\label{mean-of-steps-with-na-data-taken-care-of}

as.integer(mean(sumTable2\(Steps)) ## [1] 10821 ## Median of Steps with NA data taken care of as.integer(median(sumTable2\)Steps))
\#\# {[}1{]} 11015 \#\# Creating the histogram of total steps per day,
categorized by data set to show impact
hist(sumTable2\(Steps, breaks=5, xlab="Steps", main = "Total Steps per Day with NAs Fixed", col="Black") hist(sumTable\)Steps,
breaks=5, xlab=``Steps'', main = ``Total Steps per Day with NAs Fixed'',
col=``Grey'', add=T) legend(``topright'', c(``Imputed Data'', ``Non-NA
Data''), fill=c(``black'', ``grey'') )

The new mean of the imputed data is 10821 steps compared to the old mean
of 10766 steps. That creates a difference of 55 steps on average per
day.

The new median of the imputed data is 11015 steps compared to the old
median of 10765 steps. That creates a difference of 250 steps for the
median.

However, the overall shape of the distribution has not changed.

\subsection{Are there differences in activity patterns between weekdays
and
weekends?}\label{are-there-differences-in-activity-patterns-between-weekdays-and-weekends}

Create a new factor variable in the dataset with two levels -
``weekday'' and ``weekend'' indicating whether a given date is a weekday
or weekend day.

\subsection{Create new category based on the days of the
week}\label{create-new-category-based-on-the-days-of-the-week}

mergeData\(DayCategory <- ifelse(mergeData\)day \%in\% c(``Saturday'',
``Sunday''), ``Weekend'', ``Weekday'') Make a panel plot containing a
time series plot (i.e.~type = ``l'') of the 5-minute interval (x-axis)
and the average number of steps taken, averaged across all weekday days
or weekend days (y-axis).

library(lattice) \#\# Warning: package `lattice' was built under R
version 3.1.3 \#\# Summarize data by interval and type of day
intervalTable2 \textless{}- ddply(mergeData, .(interval, DayCategory),
summarize, Avg = mean(steps))

\subsection{Plot data in a panel plot}\label{plot-data-in-a-panel-plot}

xyplot(Avg\textasciitilde{}interval\textbar{}DayCategory,
data=intervalTable2, type=``l'', layout = c(1,2), main=``Average Steps
per Interval Based on Type of Day'', ylab=``Average Number of Steps'',
xlab=``Interval'')

Yes, the step activity trends are different based on whether the day
occurs on a weekend or not. This may be due to people having an
increased opportunity for activity beyond normal work hours for those
who work during the week.

\end{document}
